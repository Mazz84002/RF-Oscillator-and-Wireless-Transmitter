\documentclass{beamer}
\usepackage{amsmath}
\usepackage{verbatim}
\usepackage{graphicx}
\usepackage{xcolor}
\usepackage{listings}
\usepackage{tikz}
\usepackage{lipsum}

\usetheme{metropolis}           % Use metropolis theme
\title{Designing Oscillator for an Antenna at \(\sim\)3.5 GHz}

\date{}

\subtitle{2896}
\author{Mazz Shaikh(932056724), Nir Finch Cohen(230336612)}
\date{Edoh Shaulov}
\institute{Tel Aviv University}

% Define the logo image file
\newcommand{\mylogo}{\includegraphics[height=1cm]{tau.jpg}} % Adjust the height as needed
\setbeamercovered{transparent}

% Change the color of the headings
\setbeamercolor{frametitle}{bg=gray!15, fg=black} % Adjust the color as needed

% Define the background template2
\setbeamertemplate{background}{
    \begin{tikzpicture}[remember picture,overlay]
        \fill[white] (current page.south west) rectangle (current page.north east); % White background
        \node[anchor=north east, inner sep=0pt] at (current page.north east) {\mylogo};
    \end{tikzpicture}
}
\begin{document}
\maketitle

% ----------------- section 1 --------------------

\section{Milestones completed so far}

\begin{frame}{List of Milestones completed}

\begin{itemize}
  \item<1-> Created test schematic for oscillator using an ideal transistor
  \item<2-> Ran LTSpice Simulations for the ideal version
  \item<3-> Selected an RF transistor with a required performance at the interested frequency
  \item<4-> Ran PSpice Simulations for the same schematic but with non-ideal transistor
  \item<5-> Collected useful data like S-parameters, Z-parameters, \(Z_{in}\) and \(Z_{out}\) in order to build matching network
  \item<6-> Built matching network for \(Z_{out}\) and \(Z_{load}=50[\Omega]\) at \(\sim3.5[GHz]\)
  \item<7-> Tested the power output of the circuit for the matching and adjusted the values to maximize power transfer between the DC sources and \(Z_{load}\)
\end{itemize}

\end{frame}

% ----------------- section 1 --------------------

\section{Topology of Ideal Transistor Circuit}

\begin{frame}{Collpit's Oscillator}
\begin{columns}
  \column{0.5\textwidth}
  \begin{itemize}
    \item <1-> The non-feedback Collpit's version was used for better performance at high frequencies
    \item <2-> The circuit was tested with no load attached
    \item <3-> Values of \(L_p\), \(C_1\) and \(C_2\) were computed using the operating frequency formula \[f_c\approx\frac{1}{2\pi\sqrt{L_p\frac{C_1C_2}{C_1+C_2}}}\]\footnote{Full Derivation in the Appendix}
  \end{itemize}
  \column{0.5\textwidth}
\end{columns}

\end{frame}

\begin{frame}{Ouput Waveform}


\end{frame}



% ----------------- section 2 --------------------

\section{Choosing the BJT}
\begin{frame}{Reqired characteristics}
  
\end{frame}



% ----------------- section 3 --------------------

\section{Testing with Load and Choosing a Matching Network}


% ----------------- section 4 --------------------

\section{Next Steps}

\section*{Appendix}


\begin{frame}{Proof of operating frequency}

\end{frame}

\begin{frame}[fragile]{BFP520 Spice File}
\begin{tiny}
  \begin{lstlisting}
    *$
    .SUBCKT BFP520/INF 200 100 300
    L1    1   10    0.47nH
    L2    2   20    0.56nH
    L3    3   30    0.23nH
    C1   10   20    6.9fF
    C2   20   30    134fF
    C3   30   10    136fF
    L4   10  100    0.53nH
    L5   20  200    0.58nH
    L6   30  300    0.05nH
    Q1   2 1 3 BFP520
    .ENDS
    .MODEL BFP520 NPN(
    + IS =1.5E-17      NF =1            NR =1
    + ISE=2.5E-14      NE =2            ISC=2E-14
    + NC =2            BF =235          BR =1.5
    + VAF=25           VAR=2            IKF=0.4
    + IKR=0.01         RB =11           RBM=7.5
    + RE =0.6          RC =7.6          CJE=2.35E-13
    + VJE=0.958        MJE=0.335        CJC=9.3E-14
    + VJC=0.661        MJC=0.236        CJS=0
    + VJS=0.75         MJS=0.333        FC=0.5
    + XCJC=1           TF=1.7E-12       TR=5E-08
    + XTF=10           ITF=0.7          VTF=5
    + PTF=50           XTB=-0.25        XTI=0.035
    + EG=1.11)
    ***************************************************************
    *$
  \end{lstlisting}
\end{tiny}

\end{frame}

\begin{frame}{Calculation of matching network}

\end{frame}


\begin{frame}{Bibliography}

\end{frame}


\begin{comment}
\section{Section 2}
\label{sec:section2}

\begin{frame}{Title}
This is a frame in Section 2.
\end{frame}

% Add more sections and frames as needed

\begin{frame}{Table of Contents}
\tableofcontents
\end{frame}

\begin{frame}{Links to Sections}
\begin{itemize}
    \item \hyperlink{sec:section1}{Section 1}
    \item \hyperlink{sec:section2}{Section 2}
\end{itemize}
\end{frame}

\begin{frame}
  \begin{block}{Remark}
    This is a remark
  \end{block}

  \begin{example}
    This is an example
  \end{example}
  
  \begin{theorem}{Pytha}
    This is a Theorem
  \end{theorem}

\end{frame}
\end{comment}

\end{document}