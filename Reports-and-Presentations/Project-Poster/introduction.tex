Radio Frequency (RF) oscillators produce high-frequency sinusoidal waveforms, ranging from 3 kHz to 700 GHz, essential for communication systems and signal processing within transceiver architectures. Consisting of both active and passive components in a feedback setup, RF oscillators maintain resonant oscillations through gain, despite their complex non-linear design challenges. The Colpitts oscillator, a type of RF oscillator, uses an inductor and two capacitors in an LC circuit with feedback capacitors and an amplifier to generate stable, high-frequency oscillations, commonly used in receivers and frequency synthesis. It is noted for its frequency stability and low phase noise. Microstrip antennas, consisting of flat metal patches on a grounded dielectric substrate, are valued for their compact, lightweight design. They perform well at microwave frequencies, suitable for use in mobile phones, satellite communications, and radar systems. Integrated easily with printed circuit boards, these antennas work on various surfaces, offering narrow bandwidth and supporting multiple polarizations, making them versatile for communication systems.